\documentclass[11pt,journal,compsoc]{IEEEtran}

\providecommand{\PSforPDF}[1]{#1}

\newcommand\MYhyperrefoptions{bookmarks=true,bookmarksnumbered=true,
pdfpagemode={UseOutlines},plainpages=false,pdfpagelabels=true,
colorlinks=true,linkcolor={black},citecolor={black},pagecolor={black},
urlcolor={black},
pdftitle={Bare Demo of IEEEtran.cls for Computer Society Journals},%<!CHANGE!
pdfsubject={Typesetting},%<!CHANGE!
pdfauthor={Alan, Gary, and Dustin},%<!CHANGE!
pdfkeywords={Data Analytics, In-memory database, Concurrency}
}

% correct bad hyphenation here
\hyphenation{op-tical net-works semi-conduc-tor}

\begin{document}
%
% paper title
% can use linebreaks \\ within to get better formatting as desired
\title{Making Cheetah Faster}

\author{Alan~Ng,
        Gary~Chaw,
        Dustin~Kut~Moy~Cheung\vspace{5 mm}\\Department~of~Electrical~and~Computer~
    Engineering\\University~of~Toronto\\\{alan.ng,~gary.chaw,~dustin.kutmoycheung\}@mail.utoronto.ca}
\IEEEcompsoctitleabstractindextext{%
\begin{abstract}
%\boldmath
In this report, we describe how we modified Cheetah, an in-memory database, to
speed up queries and insertions using threads. The queries and inserts are
transformed into tasks in  a queue and the threads in a threadpool ‘pull’ tasks
to execute. Our results show a speedup in query time as the number of threads
increase when using the column store. However the inserts suffer from the global
lock imposed on the store which results in poor performance. Our results show
that it is possible to improve the performance of Cheetah by adding more threads
and exploiting the different cores in a multi-core processor. However we were
unable to explore the impact of cache locality in our design to determine a
better algorithm due to tools limited in the Java ecosystem to run these kinds
of analysis.
\end{abstract}
}

% make the title area
\maketitle

\IEEEdisplaynotcompsoctitleabstractindextext

\IEEEpeerreviewmaketitle

\section{Introduction}
The need for in-memory databases has exploded in recent years with the
requirement for faster data access to provide near real-time data to users.
These types of databases are challenging the dominance of traditional relational
databases in an era where multi-core processors and main memory are becoming
cheaper.


Unlike traditional databases, in-memory databases do not require optimization of
hardware for disk storage, or tweaks to the underlying operating system,
allowing for an extremely high throughput rate for writes and access using
default configurations.


Moreover, in-memory databases do not suffer from unpredictable performance
experienced by traditional databases as the seek time required to read data from
a hard-drive is eliminated.


Typically in-memory databases are employed when huge amounts of data needs to be
queried and processed in real-time. Examples might include having a real-time
feed of tweets having a specific hashtag for users of the Twitter platform.


\IEEEPARstart{T}{his} demo file is intended to serve as a ``starter file''
for IEEE Computer Society journal papers produced under \LaTeX\ using
IEEEtran.cls version 1.7 and later.
% You must have at least 2 lines in the paragraph with the drop letter
% (should never be an issue)
I wish you the best of success.

\subsection{Subsection Heading Here}
Subsection text here.

\section{Background}
Cheetah inserts data into its store by reading JSON data saved in a file. The
JSON data consists of arrays of objects, where each object consists of many
keys. Each of these keys might contain values of type: boolean, arrays, number,
strings, and objects. The keys for each object are often different from each
other, making them unsuitable to be stored in a traditional table because of the
sparseness of the data.


In a traditional database, data are organized into tables, where each table
consists of columns. Each column has a particular type associated with it. The
global configuration of tables and the column types is known as the schema of
the database.


If one would like to store all of the objects in the JSON data in a traditional
database, one of the strategies would be to save the objects into one table,
where each column will represent the keys that all the objects have. However,
this causes huge gaps in our table since most of the time only a few keys are
used in an object, causing a waste of space and processing time.


Once those JSON data are parsed, the data can be stored into the main memory
using different configurations.

\subsection{Column Store}
The column store creates a new table for each key in the object to store. The
table name is the key whose value needs to be saved. To know which value is
linked to each object, every object being saved is assigned a unique object id,
and each value is then mapped to that object id. As such, the column table has
two columns, one for the object id and one for the value.


Those columns are implemented internally by using arrays. Other data structures
are explored later in this report.


The column store performs better memory-wise compared to the other stores if the
objects being stored do not specify a value to all the keys allowed for that
object, i.e sparse data. However, searching data requires more time using that
layout because we have to scan through all the tables in the store to see if the
‘key’ table has data for a particular object id.

\section{Conclusion}
The conclusion goes here.

% use section* for acknowledgement
\ifCLASSOPTIONcompsoc
  % The Computer Society usually uses the plural form
  \section*{Acknowledgments}
\else
  % regular IEEE prefers the singular form
  \section*{Acknowledgment}
\fi

The authors would like to thank...

% Can use something like this to put references on a page
% by themselves when using endfloat and the captionsoff option.
\ifCLASSOPTIONcaptionsoff
  \newpage
\fi

% references section

% can use a bibliography generated by BibTeX as a .bbl file
% BibTeX documentation can be easily obtained at:
% http://www.ctan.org/tex-archive/biblio/bibtex/contrib/doc/
% The IEEEtran BibTeX style support page is at:
% http://www.michaelshell.org/tex/ieeetran/bibtex/
%\bibliographystyle{IEEEtran}
% argument is your BibTeX string definitions and bibliography database(s)
%\bibliography{IEEEabrv,../bib/paper}
%
% <OR> manually copy in the resultant .bbl file
% set second argument of \begin to the number of references
% (used to reserve space for the reference number labels box)
\begin{thebibliography}{1}

\bibitem{IEEEhowto:kopka}
H.~Kopka and P.~W. Daly, \emph{A Guide to {\LaTeX}}, 3rd~ed.\hskip 1em plus
  0.5em minus 0.4em\relax Harlow, England: Addison-Wesley, 1999.

\end{thebibliography}

% that's all folks
\end{document}
