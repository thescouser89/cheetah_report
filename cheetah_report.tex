\documentclass[11pt,journal,compsoc]{IEEEtran}

\providecommand{\PSforPDF}[1]{#1}

\newcommand\MYhyperrefoptions{bookmarks=true,bookmarksnumbered=true,
pdfpagemode={UseOutlines},plainpages=false,pdfpagelabels=true,
colorlinks=true,linkcolor={black},citecolor={black},pagecolor={black},
urlcolor={black},
pdftitle={Bare Demo of IEEEtran.cls for Computer Society Journals},%<!CHANGE!
pdfsubject={Typesetting},%<!CHANGE!
pdfauthor={Alan, Gary, and Dustin},%<!CHANGE!
pdfkeywords={Data Analytics, In-memory database, Concurrency}
}

% correct bad hyphenation here
\hyphenation{op-tical net-works semi-conduc-tor}
\usepackage[hidelinks]{hyperref}

\begin{document}
%
% paper title
% can use linebreaks \\ within to get better formatting as desired
\title{Making Cheetah Faster}

\author{Alan~Ng,
        Gary~Chaw,
        Dustin~Kut~Moy~Cheung\vspace{5 mm}\\Department~of~Electrical~and~Computer~
    Engineering\\University~of~Toronto\\\{alan.ng,~gary.chaw,~dustin.kutmoycheung\}@mail.utoronto.ca}
\IEEEcompsoctitleabstractindextext{%
\begin{abstract}
%\boldmath
In this report, we describe how we modified Cheetah, an in-memory database, to
speed up queries and insertions using threads. The queries and inserts are
transformed into tasks in  a queue and the threads in a threadpool ‘pull’ tasks
to execute. Our results show a speedup in query time as the number of threads
increase when using the column store. However inserts suffer from the global
lock imposed on the store which results in poor performance. Our results show
that it is possible to improve the performance of Cheetah by adding more threads
and exploiting the different cores in a multi-core processor. However we are
unable to explore the impact of cache locality in our design to due to limited
tools in the Java ecosystem.
\end{abstract}
}

% make the title area
\maketitle

\IEEEdisplaynotcompsoctitleabstractindextext

\IEEEpeerreviewmaketitle

\section{Introduction}
\IEEEPARstart{T}he need for in-memory databases has exploded in recent years with the
requirement for faster data access to provide near real-time data to users.
These types of databases are challenging the dominance of traditional relational
databases in an era where multi-core processors and main memory are becoming
cheaper.

Unlike traditional databases, in-memory databases do not require optimization of
hardware for disk storage, or tweaks to the underlying operating system,
allowing an extremely high throughput rate for writes and access using default
configurations. In-memory databases do not suffer from unpredictable performance
experienced by traditional databases as the seek time required to read data from
a hard-drive is eliminated.

Typically in-memory databases are employed when huge amounts of data need to be
queried and processed in real-time. An example is a real-time feed of tweets
having a specific hashtag for users of the Twitter platform~\cite{twitter}
which requires scanning through millions of tweets posted on Twitter.

Together with the explosion of in-memory databases is the emergence of the JSON
data format. JSON has replaced XML as the format of choice for serializing data
to be sent in between servers because of native JSON parsing abilities in the
Javascript language~\cite{json}. Unlike relational
databases, some NoSQL databases allows JSON as input into their system. Examples
include MongoDB, CouchDB and Elastic Search. The exploration of mapping JSON
data into memory and running queries in parallel still remains an interesting
new topic to research to find out if we can optimize the architecture to achieve
higher throughput.

\section{Background}


\section{Conclusion}
The conclusion goes here.

% use section* for acknowledgement
\ifCLASSOPTIONcompsoc
  % The Computer Society usually uses the plural form
  \section*{Acknowledgments}
\else
  % regular IEEE prefers the singular form
  \section*{Acknowledgment}
\fi

The authors would like to thank...

% Can use something like this to put references on a page
% by themselves when using endfloat and the captionsoff option.
\ifCLASSOPTIONcaptionsoff
  \newpage
\fi

% references section
\bibliography{cheetah_report}
\bibliographystyle{ieeetr}

% can use a bibliography generated by BibTeX as a .bbl file
% BibTeX documentation can be easily obtained at:
% http://www.ctan.org/tex-archive/biblio/bibtex/contrib/doc/
% The IEEEtran BibTeX style support page is at:
% http://www.michaelshell.org/tex/ieeetran/bibtex/
%\bibliographystyle{IEEEtran}
% argument is your BibTeX string definitions and bibliography database(s)
%\bibliography{IEEEabrv,../bib/paper}
%
% <OR> manually copy in the resultant .bbl file
% set second argument of \begin to the number of references
% (used to reserve space for the reference number labels box)

% \begin{thebibliography}{1}

% \bibitem{IEEEhowto:kopka}
% H.~Kopka and P.~W. Daly, \emph{A Guide to {\LaTeX}}, 3rd~ed.\hskip 1em plus
  % 0.5em minus 0.4em\relax Harlow, England: Addison-Wesley, 1999.

% \end{thebibliography}

% that's all folks
\end{document}
